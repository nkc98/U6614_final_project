% Options for packages loaded elsewhere
\PassOptionsToPackage{unicode}{hyperref}
\PassOptionsToPackage{hyphens}{url}
%
\documentclass[
  ignorenonframetext,
]{beamer}
\usepackage{pgfpages}
\setbeamertemplate{caption}[numbered]
\setbeamertemplate{caption label separator}{: }
\setbeamercolor{caption name}{fg=normal text.fg}
\beamertemplatenavigationsymbolsempty
% Prevent slide breaks in the middle of a paragraph
\widowpenalties 1 10000
\raggedbottom
\setbeamertemplate{part page}{
  \centering
  \begin{beamercolorbox}[sep=16pt,center]{part title}
    \usebeamerfont{part title}\insertpart\par
  \end{beamercolorbox}
}
\setbeamertemplate{section page}{
  \centering
  \begin{beamercolorbox}[sep=12pt,center]{part title}
    \usebeamerfont{section title}\insertsection\par
  \end{beamercolorbox}
}
\setbeamertemplate{subsection page}{
  \centering
  \begin{beamercolorbox}[sep=8pt,center]{part title}
    \usebeamerfont{subsection title}\insertsubsection\par
  \end{beamercolorbox}
}
\AtBeginPart{
  \frame{\partpage}
}
\AtBeginSection{
  \ifbibliography
  \else
    \frame{\sectionpage}
  \fi
}
\AtBeginSubsection{
  \frame{\subsectionpage}
}
\usepackage{amsmath,amssymb}
\usepackage{lmodern}
\usepackage{iftex}
\ifPDFTeX
  \usepackage[T1]{fontenc}
  \usepackage[utf8]{inputenc}
  \usepackage{textcomp} % provide euro and other symbols
\else % if luatex or xetex
  \usepackage{unicode-math}
  \defaultfontfeatures{Scale=MatchLowercase}
  \defaultfontfeatures[\rmfamily]{Ligatures=TeX,Scale=1}
\fi
% Use upquote if available, for straight quotes in verbatim environments
\IfFileExists{upquote.sty}{\usepackage{upquote}}{}
\IfFileExists{microtype.sty}{% use microtype if available
  \usepackage[]{microtype}
  \UseMicrotypeSet[protrusion]{basicmath} % disable protrusion for tt fonts
}{}
\makeatletter
\@ifundefined{KOMAClassName}{% if non-KOMA class
  \IfFileExists{parskip.sty}{%
    \usepackage{parskip}
  }{% else
    \setlength{\parindent}{0pt}
    \setlength{\parskip}{6pt plus 2pt minus 1pt}}
}{% if KOMA class
  \KOMAoptions{parskip=half}}
\makeatother
\usepackage{xcolor}
\newif\ifbibliography
\usepackage{longtable,booktabs,array}
\usepackage{calc} % for calculating minipage widths
\usepackage{caption}
% Make caption package work with longtable
\makeatletter
\def\fnum@table{\tablename~\thetable}
\makeatother
\usepackage{graphicx}
\makeatletter
\def\maxwidth{\ifdim\Gin@nat@width>\linewidth\linewidth\else\Gin@nat@width\fi}
\def\maxheight{\ifdim\Gin@nat@height>\textheight\textheight\else\Gin@nat@height\fi}
\makeatother
% Scale images if necessary, so that they will not overflow the page
% margins by default, and it is still possible to overwrite the defaults
% using explicit options in \includegraphics[width, height, ...]{}
\setkeys{Gin}{width=\maxwidth,height=\maxheight,keepaspectratio}
% Set default figure placement to htbp
\makeatletter
\def\fps@figure{htbp}
\makeatother
\setlength{\emergencystretch}{3em} % prevent overfull lines
\providecommand{\tightlist}{%
  \setlength{\itemsep}{0pt}\setlength{\parskip}{0pt}}
\setcounter{secnumdepth}{-\maxdimen} % remove section numbering
\ifLuaTeX
  \usepackage{selnolig}  % disable illegal ligatures
\fi
\IfFileExists{bookmark.sty}{\usepackage{bookmark}}{\usepackage{hyperref}}
\IfFileExists{xurl.sty}{\usepackage{xurl}}{} % add URL line breaks if available
\urlstyle{same} % disable monospaced font for URLs
\hypersetup{
  pdftitle={Did Medicaid Expansion Affect Drug Overdose Deaths in Appalachia?},
  pdfauthor={Neil Chin (nkc2124) \& Nicole Yahui Wu (yw3551)},
  hidelinks,
  pdfcreator={LaTeX via pandoc}}

\title{Did Medicaid Expansion Affect Drug Overdose Deaths in
Appalachia?}
\author{Neil Chin (nkc2124) \& Nicole Yahui Wu (yw3551)}
\date{2022-12-05}

\begin{document}
\frame{\titlepage}

\begin{frame}{Research Motivations}
\protect\hypertarget{research-motivations}{}
\begin{itemize}
\tightlist
\item
  Deaths due to drug overdose are a pressing problem facing United
  States policymakers and society at large, with reportedly more than
  932,000 people dying due to overdose since 1999
  (\href{https://www.cdc.gov/drugoverdose/deaths/index.html}{CDC,
  2022}).
\item
  Rates of overdose deaths increased in nearly every US state from
  2013-2017, with particularly severe incidence in the Appalachian
  region
  (\href{https://www.cdc.gov/drugoverdose/deaths/2013-2017-increase.html}{CDC,
  2020}; \href{https://www.cdc.gov/drugoverdose/deaths/2014.html}{CDC,
  2021}).
\item
  The increase in deaths is driven primarily by the ongoing US opioid
  crisis, with the vast majority (\textgreater80\%) of overdose deaths
  associated with opioid use
  (\href{https://www.cdc.gov/drugoverdose/deaths/index.html}{CDC,
  2022}).
\item
  \textbf{In this context, we aim to assess whether expansion of the
  social safety net through government policy can be causally linked to
  reductions in drug overdose death rates.}
\end{itemize}
\end{frame}

\begin{frame}{Policy Background}
\protect\hypertarget{policy-background}{}
\begin{itemize}
\tightlist
\item
  The Affordable Care Act (ACA) was passed by the United States Congress
  and signed into law by President Barack Obama in 2010, drastically
  changing the policy landscape for health care in the United States.
\item
  \textbf{Among the major provisions in the ACA was expanded eligibility
  for Medicaid (i.e., ``Medicaid Expansion''), which allowed states the
  option to raise the income-eligibility threshold to 138\% of the
  federal poverty level
  (\href{https://www.kff.org/medicaid/issue-brief/status-of-state-medicaid-expansion-decisions-interactive-map/}{KFF,
  2022}).}
\item
  Of the \textbf{13 states} whose boundaries overlap with the broad
  geographical definition of Appalachia, \textbf{five states} (Kentucky,
  Maryland, New York, Ohio, and West Virginia) passed legislation
  mandating the expansion of Medicaid as of January 1st, 2014
  (\href{https://www.kff.org/medicaid/issue-brief/status-of-state-medicaid-expansion-decisions-interactive-map/}{KFF,
  2022}). \textbf{Two additional states}, Pennsylvania and Virginia,
  would later expand Medicaid, with the former in 2015 and the latter in
  2019.
\item
  \textbf{Six states} (Alabama, Georgia, Mississippi, North Carolina,
  South Carolina, and Tennessee) have not expanded Medicaid to-date.
\end{itemize}
\end{frame}

\begin{frame}{Policy Background}
\protect\hypertarget{policy-background-1}{}
\begin{itemize}
\tightlist
\item
  In total, 210 Appalachian counties are located in expansion states and
  213 Appalachian counties are located in non-expansion states.
\end{itemize}

\includegraphics{project_slides_files/figure-beamer/display map-1.pdf}
\end{frame}

\begin{frame}{Research Hypothesis}
\protect\hypertarget{research-hypothesis}{}
\begin{itemize}
\tightlist
\item
  In our study, we identify the discrepancy in Medicaid expansion as a
  policy ``treatment'' (i.e., ``differential exposure between entities
  over time'') affecting drug overdose incidence in Appalachian
  communities.
\item
  Our \textbf{\emph{hypothesized}} causal mechanism is that expanded
  Medicaid eligibility allowed for greater access to low-cost health
  care among Appalachian counties in expansion states, therefore
  enabling people struggling with drug addiction to receive treatment,
  \textbf{\emph{reducing}} overall deaths from drug overdose in these
  areas.
\item
  Existing literature has also examined the counter-hypothesis that
  Medicaid expansion may have actually \textbf{\emph{increased}} drug
  overdose rates by increasing access to prescription opioids.
  \href{https://onlinelibrary.wiley.com/doi/epdf/10.1111/add.14741}{Swartz
  and Beltran (2019)} find that, while Medicaid expansion did increase
  prescription opioid availability, there was no accompanying increase
  in overdose mortality.
  \href{https://www.ncbi.nlm.nih.gov/pmc/articles/PMC6318168/}{Venkataramani
  and Chatterjee (2018)} examine early 2000s Medicaid expansion in
  Arizona, Maine, and New York, and find that expansion did in fact
  \textbf{\emph{decrease}} overdose death rates relative to neighboring
  non-expansion states.
\end{itemize}
\end{frame}

\begin{frame}{Data Description}
\protect\hypertarget{data-description}{}
\begin{itemize}
\tightlist
\item
  Our data is a county-year panel dataset, comprised of Appalachian
  counties over the period 2011-2019. For analysis, we restrict the
  panel to the three years prior to Medicaid expansion and the five
  years after, resulting in a final dataset of 3384 observations.
\item
  Data on drug overdose death rates (i.e., deaths per 100,000 residents)
  comes from estimates modeled by the National Center for Health
  Statistics (NCHS), which are available at the county-level for the
  period 2003-2020.
\item
  County-level demographic covariates are taken from the US Census
  Bureau American Community Survey (ACS).
\end{itemize}

\begin{longtable}[]{@{}lrrlllr@{}}
\toprule()
COUNTY & Year & FIPS & STATE & medicaid\_expansion & expansion\_date &
treat \\
\midrule()
\endhead
Adair & 2011 & 21001 & Kentucky & Expansion & 2014-01-01 & 0 \\
Adair & 2012 & 21001 & Kentucky & Expansion & 2014-01-01 & 0 \\
Adair & 2013 & 21001 & Kentucky & Expansion & 2014-01-01 & 0 \\
Adair & 2014 & 21001 & Kentucky & Expansion & 2014-01-01 & 1 \\
\bottomrule()
\end{longtable}
\end{frame}

\begin{frame}{Data Description}
\protect\hypertarget{data-description-1}{}
\begin{itemize}
\tightlist
\item
  Exploring the balance across key continuous variables between counties
  in expansion states and counties in non-expansion states at the time
  when Medicaid expansion occurs (i.e., Jan.~1st, 2014 for all states
  expect PA, Jan 1st, 2015 for PA), we find that observable
  characteristics of the expansion and non-expansion counties are
  broadly similar.
\end{itemize}

\begin{table}

\caption{\label{tab:create diif in means}Difference-in-means during year of Medicaid expansion by county expansion status}
\centering
\resizebox{\linewidth}{!}{
\begin{tabular}[t]{lrrrrrr}
\toprule
\multicolumn{1}{c}{ } & \multicolumn{2}{c}{Non-Expansion (N=213)} & \multicolumn{2}{c}{Expansion (N=210)} & \multicolumn{2}{c}{ } \\
\cmidrule(l{3pt}r{3pt}){2-3} \cmidrule(l{3pt}r{3pt}){4-5}
  & Mean & Std. Dev. & Mean & Std. Dev. & Diff. in Means & p\\
\midrule
Median Income (\$) & 39623.55 & 7786.51 & 40245.98 & 8272.65 & 622.43 & 0.43\\
Median Age & 41.54 & 4.18 & 42.05 & 3.26 & 0.50 & 0.17\\
Male Share & 49.19 & 1.35 & 49.97 & 2.32 & 0.78 & <0.001\\
Black Share & 10.62 & 14.52 & 2.49 & 3.05 & -8.14 & <0.001\\
Hispanic Share & 4.09 & 4.14 & 1.60 & 1.66 & -2.48 & <0.001\\
White Share & 82.72 & 14.76 & 93.89 & 4.81 & 11.17 & <0.001\\
Asian Share & 0.71 & 1.12 & 0.57 & 0.94 & -0.14 & 0.16\\
\bottomrule
\multicolumn{7}{l}{\rule{0pt}{1em}Note: Observations are weighted by the population in each county.}\\
\end{tabular}}
\end{table}
\end{frame}

\begin{frame}{Exploratory Analaysis}
\protect\hypertarget{exploratory-analaysis}{}
Examining overdose death rates over time for expansion and non-expansion
counties in Appalachia, it appears that Medicaid expansion had a
\textbf{\emph{positive}} effect on overdose death rates, contradicting
the direction of our hypothesized causal effect.

\includegraphics{project_slides_files/figure-beamer/display line plot-1.pdf}
\end{frame}

\begin{frame}{Empirical Strategy}
\protect\hypertarget{empirical-strategy}{}
To evaluate the effect of Medicaid expansion on drug overdose deaths in
Appalachian counties, we estimate the following
``differences-in-differences'' specification:

\[Overdose~Death~Rate_{it} = \beta Medicaid~Expansion_{it} + \textbf{X}_{it} \gamma + \nu_{i} + \tau_{t} + \varepsilon_{it}\]

where \(Overdose~Death~Rate_{it}\) is deaths attributed to drug overdose
per 100,000 county residents for county \(i\) at time \(t\),
\(Medicaid~Expansion_{it}\) indicates ``treatment'' status (i.e.,
enactment of Medicaid expansion) for a county-year, \(\textbf{X}_{it}\)
is a vector of time varying controls (e.g., median income, median age,
male population share, racial composition) for potential county-level
determinants of overdose death rates outside of our policy variation of
interest.

Additionally, we include an array of county fixed effects, \(\nu_{i}\),
that control for unobserved time-invariant factors specific to
individual counties. We further include \(\tau_{t}\), year fixed
effects, to control for unobserved county-invariant factors that might
have changed between each year included in our panel. Finally,
\(\varepsilon_{it}\) is the idiosyncratic error term.
\end{frame}

\begin{frame}{Estimation Results}
\protect\hypertarget{estimation-results}{}
We test two model specifications to estimate difference-in-difference
effects. The first considers effects up to two years following Medicaid
expansion, while the second considers effects up to five years following
Medicaid expansion.

\begin{table}[!h]

\caption{\label{tab:estimate models}Effect of Medicaid Expansion on Drug Overdose Death Rates}
\centering
\resizebox{\linewidth}{!}{
\begin{tabular}[t]{lcc}
\toprule
  & Two Years Post-Expansion & Five Years Post-Expansion\\
\midrule
Medicaid Expansion & \num{4.0803}*** & \num{7.8280}***\\
 & (\num{0.6920}) & (\num{1.1228})\\
Median Income (\$) & \num{-0.0004}*** & \num{-0.0004}***\\
 & (\num{0.0001}) & \vphantom{1} (\num{0.0001})\\
Median Age & \num{0.0365} & \num{0.0113}\\
 & (\num{0.2306}) & (\num{0.2591})\\
Male Population Share (0-100) & \num{0.0000} & \num{0.0001}*\\
 & (\num{0.0001}) & (\num{0.0000})\\
Black Population Share & \num{-0.0001}* & \num{-0.0001}**\\
 & (\num{0.0001}) & (\num{0.0001})\\
Hispanic Population Share & \num{-0.0002} & \num{-0.0001}\\
 & (\num{0.0002}) & (\num{0.0001})\\
Asian Population Share & \num{0.0009}*** & \num{0.0005}\\
 & (\num{0.0003}) & (\num{0.0003})\\
\midrule
N & \num{2115} & \num{3384}\\
R-squared & \num{0.779} & \num{0.766}\\
Adj. R-squared & \num{0.738} & \num{0.741}\\
County FEs & X & X\\
Year FEs & X & X\\
\bottomrule
\multicolumn{3}{l}{\rule{0pt}{1em}Robust standard errors clustered by county are shown in parentheses.
                      Observations are weighted by the population in each county.}\\
\multicolumn{3}{l}{\rule{0pt}{1em}* p $<$ 0.1, ** p $<$ 0.05, *** p $<$ 0.01}\\
\end{tabular}}
\end{table}
\end{frame}

\begin{frame}{Discussion}
\protect\hypertarget{discussion}{}
\begin{itemize}
\item
  Our estimates provide significant evidence for the conclusion that
  Medicaid expansion in Appalachian states had the unintended
  consequence of \textbf{\emph{increasing}} deaths from drug overdose. A
  possible mechanism for this result is increased access to prescription
  opioids.
\item
  Our findings differ from previous studies on the effect of Medicaid
  expansion on overdose deaths. However, one of these studies
  \href{https://onlinelibrary.wiley.com/doi/epdf/10.1111/add.14741}{Swartz
  and Beltran (2019)}, did find that expansion increased the supply of
  prescription opioids and that there could be a lagged increase in
  overdose deaths.
\item
  Key limitations include potential OVB due to county-year variation in
  number of Substance Use Disorder (SUD) treatment facilities and lack
  of external validity outside of Appalachia. The latter limitation
  could explain why our results differ from previous studies.
\end{itemize}
\end{frame}

\end{document}
